\documentclass[]{article}
\usepackage{color}
\usepackage{hologo}
\usepackage{enumitem}
\newcommand{\BlueText}[1]{\textcolor{blue}{#1}}


\begin{document}
\noindent \BlueText{\textbf{Exercise 5:} Attempt to replicate the following in black text by following these steps.} 
\vspace*{0.5cm}

\BlueText{
\begin{enumerate}[label=\textbf{Step \arabic*:}]
	\item Create a \hologo{BibTeX} file that includes 4 random citations.
	You can pull the citations from Google Scholar.
	\item Replicate the following sentence randomly citing 3 of these 4 references.
	\item Make sure to include your bibliography at the end using \texttt{ieeetr} bibliography style.
	\item Once everything has been replicated try including the \texttt{cite} package in your preamble.
	What changed when you included this package?
	\item Now try changing the bibliography style to \texttt{apalike}.
	What changed?
\end{enumerate}
}

%%%%%%%%%%%%%%%%
%%%% DONE: Add the text to see reference list
From \cite{phillips2006invasion}, I found out that an alternative understanding of concept of XYZ \cite{davies1978deep} was proposed in the book \cite{mattison1987frogs}.
From attending the MEGC \LaTeX\ workshop, I now feel comfortable using \hologo{BibTeX}.
I even eel comfortable enough trying something new that has not been taught.
For example I can now cite multiple citations at once \cite{phillips2006invasion, davies1978deep, mattison1987frogs}.

\vspace{1cm}

\bibliographystyle{ieeetr}
\bibliography{Exercise5Bib}

\noindent \BlueText{\textbf{\underline{NOTE:}} Since the third reference is a book, the reference like has a different style than the two articles above.\\\\
Also notice that the 4th entry of the \texttt{.bib} file doesn't show up in the reference section since only referenced entries will be included in the References section.}

\end{document}